\chapter{Η Υλοποίηση}
\label{ch:Implementation}

\newcommand{\myTTT}[1]{\texttt{\hyphenchar \font=`\- \spaceskip=.5em plus .5em \xspaceskip=.5em #1}}

Υλοποιήσαμε όλους τους αλγορίθμους που αναφέραμε στο Κεφάλαιο \ref{ch:Algorithms} και σχεδιάσαμε μια γραφική διαπροσωπεία για την οπτικοποίηση της εισόδου και των
αποτελεσμάτων. Η υλοποίηση έγινε στη γλώσσα προγραμματισμού \textbf{Python 3} και η οπτικοποίηση με την βιβλιοθήκη \textbf{PyVis}\cite{pyvis-docs}

\section{Οργάνωση Κώδικα}

Η γενική αρχιτεκτονική για την επίλυση του κάθε αλγορίθμου είναι η ίδια σε κάθε περίπτωση. Πιο συγκεκριμένα η υλοποίηση οργανώνεται στα αρχεία: 
\begin{itemize}
	\item \myTTT{main.py} : η αρχή του εκτελέσιμου προγράμματος.
	
	\item \myTTT{graph.py} : η διευθύνουσα κλάση του γραφήματος.
	
	\item \myTTT{polynomial\_time\_algorithm.py} : η διευθύνουσα κλάση για την υλοποίηση του αλγορίθμου.
	 
\end{itemize} 

Επιπλέον αρχεία υλοποιούν κλάσεις οι οποίες είναι απαραίτητες για την υλοποίηση των επιμέρους αλγορίθμων. Αναλυτικά θα τις αναλύσουμε στις αντίστοιχες ενότητες. 

\section{Είσοδος - Έξοδος}

Κάθε αλγόριθμος παίρνει σαν είσοδο ένα AT-free γράφημα. Η αναπαράσταση του γραφήματος έχει γίνει σε αρχείο JSON καθώς η διαχείριση του είναι αρκετά εύκολη από γλώσσες προγραμματισμού όπως η \textbf{Python}. 

Το αρχείο JSON έχει τα κλειδιά "num\_of\_vertices" και "num\_of\_edges" τα οποία δηλώνουν τον αριθμό των κόμβος του γραφήματος και ακμών αντίστοιχα. Οι τιμές που παίρνουν είναι θετική ακέραιοι $\mathbb{N}$. Επιπλέον υπάρχει το κλειδί "edges" το οποίο έχει όλες τις ακμές του γραφήματος. Η δήλωση των ακμών γίνεται με έναν δισδιάστατο πίνακα που περιέχει ζεύγη ακεραίων που υποδηλώνουν τους συνδεδεμένους κόμβους.

Το αρχείο υφίσταται μια ανάλυση προκειμένου να επιβεβαιωθεί η απουσία AT (asteroidal triple). Αυτή η ανάλυση δεν είναι η βέλτιστη αλλά μας διασφαλίζει ότι το γράφημα είναι AT-free. Η κλάσει για την ανάλυση του αρχείου:

\lstinputlisting[
language=Python,
breaklines=true,
basicstyle=\footnotesize\ttfamily,
columns=fullflexible,
breakatwhitespace=false,
captionpos=t,
caption={graph\_parser.py}
]{Content/CodeDirectory/graph_parser.py}



Αν η ανάλυση του γραφήματος είναι επιτυχημένη, τότε δημιουργείται ένα αντικείμενο τύπου Graph. Παρακάτω παρουσιάζεται ο ορισμός του Graph, καθώς και τα πεδία που δημιουργούνται. Τα συγκεκριμένα πεδία είναι κοινά και για τους τρεις αλγορίθμους που έχουμε υλοποιήσει. 

  

\begin{lstlisting}[
	language=Python,
	breaklines=true,
	basicstyle=\footnotesize\ttfamily,
	columns=fullflexible,
	breakatwhitespace=false,
	captionpos=t,
	caption={Ορισμός κλάσης Graph},
	label={lst:graph-compiler}
	]
	class Graph:
		def __init__(self, num_of_vertices, 	num_of_edges, edges, vertices=None):
			self.num_of_vertices = num_of_vertices
			self.num_of_edges = num_of_edges
			self.vertices = set(str(i) for i in range(num_of_vertices)) if vertices is None else vertices
			self.edges = set((str(u), str(v)) for u, v in edges)
			self.adjacency_list = {str(vertex): set() for vertex in self.vertices}
			for edge in self.edges:
				u, v = edge
				self.adjacency_list[u].add(v)
				self.adjacency_list[v].add(u)
\end{lstlisting}


Η έξοδος στον αλγόριθμο υπολογισμού μέγιστου ανεξάρτητου συνόλου
είναι ο αριθμός ανεξαρτησίας του γραφήματος καθώς και ένα σετ από κόμβους που είναι ένα μέγιστο ανεξάρτητο σύνολο. 


Ο αλγόριθμος υπολογισμού του ελάχιστου κυρίαρχου συνόλου επιστρέφει ένα σετ κόμβων που αποτελεί ένα ελάχιστο κυρίαρχο σύνολο.  

Για το πρόβλημα του 3-χρωματισμού, η έξοδος του αλγορίθμου παρουσιάζεται σε μορφή λεξικού, όπου κάθε κόμβος του γραφήματος αντιστοιχεί σε ένα κλειδί, και η σχετική τιμή είναι το χρώμα που έχει ανατεθεί σε αυτόν τον κόμβο (π.χ., "red", "green", "blue").

Σε όλους τους αλγορίθμους εκτελούμε την συνάρτηση \myTTT{show()} της κλάσης Graph. Η συνάρτηση επιστρέφει ένα αρχείο HTML το οποίο περιέχει την οπτικοποίηση του γραφήματος.

\section{Δομές Δεδομένων}


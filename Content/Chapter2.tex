\chapter{Οι Αλγόριθμοι}
\label{ch:Algorithms}

Σε αυτό το κεφάλαιο θα μελετήσουμε με λεπτομέρεια όλους τους αλγορίθμους που υλοποιήσαμε για τα AT-free γραφήματα.
Για κάθε αλγόριθμο θα δώσουμε τον συμβολισμό και τα λήμματα που χριάζονται για την περιγραφή του. Αμέσως μετά, θα εξηγήσουμε την πολυπλοκότητά του και συγκεκτριμένα βήματά του, που θεωρήσαμε πιο ιδιαίτερα. Τέλος θα δώσουμε και ένα απλό παράδειγμα επίλυσης του κάθε αλγορίθμου.

Διευκρινίζουμε ότι δεν αποδεικνύουμε την ορθότητα του κάθε αλγορίθμου που χρησιμοποιούμε. Οι αποδείξεις αυτές βρίσκονται στις αντίστοιχες αναφορές(\cite{at-free-independent-sets}, \cite{at-free-domination}, \cite{at-free-3-colouring})   

% -------------
% SECTION START
% -------------
\section{Υπολογισμός Μέγιστου Ανεξάρτητου Συνόλου}
\label{sec:Independent_Set_Alg}
Συμβολίζουμε τον αριθμό των κορυφών ενός γραφήματος $G = (V, E)$ με
$n$ και τον αριθμό των ακμών με $m$. 
Υπενθυμίζουμε ότι ένα ανεξάρτητο σύνολο σε ένα γράφημα $G$ είναι ένα σύνολο από ζεύγη με μη γειτονικές
κορυφές. Ο αριθμός ανεξαρτησίας ενός γραφήματος G, συμβολίζεται ως α(G) είναι το μέγεθος
του μεγαλύτερου ανεξάρτητου συνόλου στο γράφημα. Οι βασικές δομικές ιδιότητες που πρέπει να αναλύσουμε πριν την περιγραφή του αλγορίθμου είναι τα Intervals, Splitting intervals και Splitting comonents.

Σε ένα AT-free γράφημα $G$ όπου το $x$ και το $y$ είναι δύο ξεχωριστές μη γειτονικές κορυφές του G. Συμβολίζουμε με $C^{x}(y)$ το component του $G - N[x]$ όπου εμπεριέχεται το $y$, και με $r(x)$ τον αριθμό των components του $G - N[x]$


\section{Υπολογισμός Ελάχιστου Κυρίαρχου Συνόλου}
\label{sec:Domination_Set_Alg}


% SECTION START
% -------------
\section{Το Πρόβλημα του 3-Χρωματισμού}
\label{sec:3-Coloring_Alg}

 









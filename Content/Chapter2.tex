\chapter{Οι Αλγόριθμοι}
\label{ch:Algorithms}

Σε αυτό το κεφάλαιο θα μελετήσουμε με λεπτομέρεια όλους τους αλγορίθμους που υλοποιήσαμε για τα AT-free γραφήματα.
Για κάθε αλγόριθμο θα δώσουμε τον συμβολισμό και τα λήμματα που χρειάζονται για την περιγραφή του. Αμέσως μετά, θα εξηγήσουμε την πολυπλοκότητά του και συγκεκριμένα βήματά του, που θεωρήσαμε πιο ιδιαίτερα. Τέλος θα δώσουμε και ένα απλό παράδειγμα επίλυσης του κάθε αλγορίθμου.

Διευκρινίζουμε ότι δεν αποδεικνύουμε την ορθότητα του κάθε αλγορίθμου που χρησιμοποιούμε. Οι αποδείξεις αυτές βρίσκονται στις αντίστοιχες αναφορές(\cite{at-free-independent-sets}, \cite{at-free-domination}, \cite{at-free-3-colouring})   

% -------------
% SECTION START
% -------------
\section{Υπολογισμός Μέγιστου Ανεξάρτητου Συνόλου}
\label{sec:Independent_Set_Alg}
Συμβολίζουμε τον αριθμό των κορυφών ενός γραφήματος $G = (V, E)$ με
$n$ και τον αριθμό των ακμών με $m$. 
Υπενθυμίζουμε ότι ένα ανεξάρτητο σύνολο σε ένα γράφημα $G$ είναι ένα σύνολο από ζεύγη με μη γειτονικές
κορυφές. Ο αριθμός ανεξαρτησίας ενός γραφήματος $G$, συμβολίζεται ως $\alpha(G)$ είναι το μέγεθος
του μεγαλύτερου ανεξάρτητου συνόλου στο γράφημα. Οι βασικές δομικές ιδιότητες που πρέπει να αναλύσουμε πριν την περιγραφή του αλγορίθμου είναι τα $Components$ και τα $Intervals$.


Σε ένα AT-free γράφημα $G$ όπου το $x$ και το $y$ είναι δύο ξεχωριστές μη γειτονικές κορυφές του G. Συμβολίζουμε με $C^{x}(y)$ το $Component$ του $G - N[x]$ όπου εμπεριέχεται το $y$, και με $r(x)$ τον αριθμό των $Components$ του $G - N[x]$.

\begin{definition}
	Μια κορυφή $z \in V \setminus \{x, y\}$ είναι μεταξύ των κορυφών $x$ και $y$ εάν οι $x$ και $z$ βρίσκονται στον ίδιο $Component$ του $G - N[x]$. Εναλλακτικά, η κορυφή $z$ είναι μεταξύ των $x$ και $y$ στο γράφημα $G$ αν υπάρχει μονοπάτι από τον $x$ στον $z$ που αποφεύγει το $N[y]$ και μονοπάτι από τον $y$ στον $z$ που αποφεύγει το $N[x]$.
\end{definition}

\begin{definition}
	Το διάστημα $I = I(x, y)$ του $G$ είναι το σύνολο όλων των κορυφών του $G$ που βρίσκονται μεταξύ $x$ και $y$. Συνεπώς, $I(x, y) = C_x(y) \cap C_y(x)$.
\end{definition}

 Ο αλγόριθμος των Broersma, H., Kloks, T., Kratsch, D. και Müller, H. του καθορίζει τον αριθμό ανεξαρτησίας κάθε $Component$ και κάθε $Interval$ χρησιμοποιώντας τις σχέσεις που δίνονται στα Λήμματα \ref{lemma6_1}, \ref{lemma6_2} και \ref{lemma6_3}.
 
 \begin{lemma}
 	\label{lemma6_1}
 	'Έστω ότι $G = (V,E)$ είναι οποιαδήποτε γράφημα. 
 	Τότε $$\alpha(G)=1+\max_{x\in V}\left(\sum_{i=1}^{r(x)}\alpha(C_i^x)\right)$$ όπου  $C_1^x, C_2^x,...,C_r(x)^x$ τα $Compontes$ του $G - N[x]$.
 \end{lemma}

\begin{lemma}
	\label{lemma6_2}
		'Έστω ότι $G = (V,E)$ είναι ένα AT-free γράφημα. Έστω $x \in V$ και έστω $C^x$ ένα $Component$ του $G - N[x]$. Τότε $$\alpha(C^x)=1+\max_{y\in C^x}\left(\alpha(I(x,y))+\sum_{i}\alpha(D_i^y)\right)$$ όπου τα $D_i^y$ είναι $Components$ του $G - N[y]$ που εμπεριέχονται στο $C^x$. 
\end{lemma}

\begin{lemma}
	\label{lemma6_3}
	'Έστω ότι $G = (V,E)$ είναι ένα AT-free γράφημα. Έστω $I = I(x,y)$ είναι ένα $Interval$ του $G$. Αν $I = \emptyset$ τότε $\alpha(I) = 0$. Αλλιώς $$\alpha(I)=1+\max_{s\in I}\left(\alpha(I(x,s))+\alpha(I(s,y))+\sum_{i}\alpha(C_i^s)\right)$$
\end{lemma}
	
Σε αυτό το σημείο μπορούμε να δώσουμε τον αλγόριθμο για τον υπολογισμό του αριθμού ανεξαρτησίας $\alpha(G)$ για AT-free γραφήματα, ο οποίος είναι βασισμένος στον δυναμικό προγραμματισμό. 

\begin{algorithm}[H]
	\caption{Αλγόριθμος υπολογισμού αριθμού ανεξαρτησίας σε AT-free γραφήματα}
	\label{alg:DOPC}
	
	\hspace*{\algorithmicindent} \textbf{Είσοδος:} Ένα AT-free γράφημα $G$.\\
	 
	\hspace*{\algorithmicindent} \textbf{Έξοδος:} Αριθμός ανεξαρτησίας $\alpha(G)$
	
	\begin{algorithmic}[1]
		
		\STATE Για κάθε $x \in V$, υπολόγισε όλα τα $Components$ $C_1^x , C_2^x , \ldots , C_{r(x)}^x$
		\STATE Για κάθε ζευγάρι μη γειτονικών κορυφών $x$ και $y$, υπολόγισε το $Interval$ $I(x, y)$.
		\STATE Ταξινόμησε όλα τα $Components$ και τα $Intervals$ με βάση το μη-αύξοντα αριθμό κορυφών.
		\STATE Υπολόγισε τα $\alpha(C)$ και $\alpha(I)$ για κάθε $Components$ $C$ και κάθε $Interval$ $I$ με τη σειρά του Βήματος 3.
		\STATE Υπολόγισε το $\alpha(G)$.
				
	\end{algorithmic}
\end{algorithm}

\begin{definition}
	 Υπάρχει αλγόριθμος χρόνου $O(n^4)$ για τον υπολογισμό του ανεξάρτητου αριθμού ενός AT-free γραφήματος.
\end{definition}

Για την πολυπλοκότητα του αλγορίθμου μελετάμε κάθε βήμα ξεχωριστά. Το πρώτο βήμα μπορεί να υλοποιηθεί σε χρόνο $O(n(n + m))$ χρησιμοποιώντας έναν γραμμικό αλγόριθμο για τον υπολογισμό των $Components$. Το βήμα 2 υπολογίζει $Intervals$ για τις μη γειτονικές κορυφές $x$ και $y$, χρησιμοποιώντας την τομή των συνιστωσών $C_x(y)$ και $C_y(x)$. Η διαδικασία, που εκτελείται σε χρόνο O(n) για κάθε $Interval$, οδηγεί σε συνολική χρονική πολυπλοκότητα $O(n^3)$. Η υλοποίηση αξιοποιεί ένα λεξικό με αντικείμενα της κλάσης $Interval$, παρέχοντας έναν αποτελεσματικό τρόπο διαχείρισης και υπολογισμού εντός το πολύ $n^2$ $Component$ και $Intervals$.


\section{Υπολογισμός Ελάχιστου Κυρίαρχου Συνόλου}
\label{sec:Domination_Set_Alg}


% SECTION START
% -------------
\section{Το Πρόβλημα του 3-Χρωματισμού}
\label{sec:3-Coloring_Alg}

 









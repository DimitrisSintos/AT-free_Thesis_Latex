\chapter{Επίλογος}
\label{ch:Conclusion}

Στην παρούσα εργασία καταφέραμε να υλοποιήσουμε τρεις κλασικούς αλγορίθμους της θεωρίας γραφημάτων για 
τα AT-free γραφήματα. Συγκεκριμένα μελετήσαμε τον αλγόριθμο των Ηajo Βroersma , Τon Κloks , Dieter Kratsch , και Ηaiko Μüller\cite{at-free-independent-sets} για τον υπολογισμό του ανεξάρτητου αριθμού τον οποίο επεκτείναμε ώστε να επιστρέφει και ένα μέγιστο ανεξάρτητο σύνολο, τον αλγόριθμο του Dieter Kratsch\cite{at-free-domination} για τον εντοπισμό ελάχιστου κυρίαρχου συνόλου και τον αλγόριθμο του Juraj Stacho\cite{at-free-3-colouring}  για την αντιμετώπιση του προβλήματος του 3-χρωματισμού σε ΑΤ-free γραφήματα.

Για το Πρόβλημα του 3-Χρωματισμού χρειάστηκε να υλοποιήσουμε και τον αλγόριθμο ”Biconnectivity” του Robert Tarjan \cite{tarjan-depth-first-search}.

Δεν έχουμε υλοποιήσει τον αλγόριθμο εντοπισμού ενός κυρίαρχο συντομότερου
μονοπατιού \cite{corneil-olariu-stewart-asteroidal-triple-free-graphs}, που χρειάζεται ώστε ο αλγόριθμος του Dieter Kratsch\cite{at-free-domination}, να έχει χρονική πολυπλοκότητα $O(n^6)$. Αυτό κάνει τον αλγόριθμο που έχουμε υλοποιήσει να τρέχει σε $O(n^7)$ καθώς χρειάζεται να γίνει ο έλεγχος για όλους τους κόμβους του γραφήματος. 

Για τους αλγορίθμους υπολογισμού μέγιστου ανεξάρτητου συνόλου και ελάχιστου κυρίαρχου συνόλου έχουμε υλοποιήσει αλγόριθμους "Brute force" που ελέγχουν την ορθότητα των αλγορίθμων που έχουμε υλοποιήσει. 
Για το πρόβλημα του 3-χρωματισμού υλοποιήσαμε μια συνάρτηση που ελέγχει την γειτονιά κάθε κόμβου ώστε να μην έχει το ίδιο χρώμα.

Καταφέραμε να υλοποιήσουμε όλους τους αλγορίθμους, μέσα στα όρια της πολυπλοκότητας τους, όπως αυτή περιγράφεται στις αντίστοιχες αναφορές. Παρόλα αυτά δεν έχουμε υλοποιήσει πιστοποιητικά που ελέγχουν την χρονική και χωρική πολυπλοκότητα.     
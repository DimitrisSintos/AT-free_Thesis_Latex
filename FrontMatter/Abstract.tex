\chapter*{\abstractname}
\addstarredchapter{\abstractname} % minitoc
\makecseabstract


\noindent Αστεροειδής τριάδα (ΑΤ σύντομα) είναι ένα σύνολο τριών κορυφών ενός γραφήματος τέτοιο ώστε να υπάρχει μονοπάτι μεταξύ οποιωνδήποτε δύο από αυτές αποφεύγοντας τη γειτονιά της τρίτης. Τα γραφήματα που δεν περιέχουν αστεροιειδή τριάδα ονομάζονται ΑT-free. Η κατηγορία των AT-free γραφημάτων είναι ένας τύπος γραφήματος για τον οποίο πολλά προβλήματα που είναι NP-πλήρη σε γενικότερα γραφήματα μπορούν να λυθούν σε πολυωνυμικό χρόνο.

Σε αυτή τη διπλωματική εργασία έχουμε υλοποιήσει τρεις αλγορίθμους πολυωνυμικού χρόνου για τα AT-free γραφήματα.

Συγκεκριμένα, τον αλγόριθμο υπολογισμού μέγιστου ανεξάρτητου συνόλου των Ηajo Βroersma, Τon Κloks, Dieter Kratsch, 
και Ηaiko Μüller\cite{at-free-independent-sets}, τον αλγόριθμο υπολογισμού Ελάχιστου Κυρίαρχου Συνόλου του Dieter Kratsch\cite{at-free-domination} και τον αλγόριθμο για το πρόβλημα του 3-Χρωματισμού του Juraj Stacho\cite{at-free-3-colouring}. 
